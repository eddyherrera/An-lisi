%--------------------------------------------------------------------
% Formato para talleres y quices
% Eddy Herrera Daza M.Sc.
% herrera.eddy@gmail.com
%--------------------------------------------------------------------
%--------------------------------------------------------------------
\documentclass[12pt,letterpaper]{exam}
\usepackage[left=2cm,top=2cm,right=2cm,bottom=4cm]{geometry}
\usepackage{hyperref}
\usepackage[utf8]{inputenc}
\usepackage[spanish,activeacute]{babel} % Escribir en espanol
\decimalpoint
\usepackage{enumerate}
\usepackage{eurosym}
\usepackage{latexsym,amsmath,amsthm,amssymb,amsfonts,bbm, dsfont}
\usepackage[mathscr]{euscript}
\usepackage{ae,aecompl}
\usepackage{graphicx}
\usepackage{fancybox}
\DeclareGraphicsExtensions{.pdf,.png,.jpg} %solo para PDFLaTeX
\usepackage{float} %para tener manejo de ubicacion de tablas y graficas
\usepackage{subfigure} %varias figura en una
% Paquete para generar varias columnas y filas
\usepackage{multicol}
\usepackage{multirow}
\usepackage[usenames]{color}
\usepackage{colortbl}
%----modificar las caption-----
\usepackage[font=small,labelfont=small,labelfont=bf,textfont=it]{caption}
%--------------------------------------------------------------------
%\newcommand{\base}[1]{\underline{\hspace{#1}}}
%--------------------------------------------------------------------
\newcommand{\uni}{Análisis Numérico}
\newcommand{\fac}{Interpolación}
\newcommand{\dep}{Eddy Herrera Daza}
%\newcommand{\mat}{Análisis Numérico } %Materia
%\newcommand{\tema}{Parcial I } %Tipo y Número de Quiz
%\newcommand{\autor}{Eddy Herrera Daza}
%\newcommand{\fecha}{Agosto 2018}
\newcommand{\espacio}[1]{\vspace{#1}}
%---------------------------------------------------------------------
%\boxedpoints
%\pointname{}
%\bonuspointname{(bono)}
%\extrawidth{0.8in}
%%\extrafootheight{1.25ino
%\extraheadheight{-0.1in}
%\pagestyle{headandfoot}
%\footrule
%\headrule
%\firstpageheader{}{}{}
%\firstpagefooter{}{\thepage $\,$ de \numpages}{}
%\runningfooter{\uni}{\thepage $\,$ de \numpages}{}

\setlength{\columnsep}{1.5cm}
\setlength{\columnseprule}{0.5pt}   % default=0pt (no line)

%---------------------------------------------------------------------------
\begin{document}
\begin{tabular}{lr}
    \multirow{2}{*} {\includegraphics[height=1.0cm]{logo}}
    	& \hspace{1.5 cm} {\textbf{\uni}} \\	
    	& {\textbf{\fac}} \\
   	& {\textbf{\dep}} 
%    	& {\textbf{\mat: \tema}} \\
%    	& {\textit{\autor}}\\
%		& {\textit{\fecha}}
\end{tabular}\\\\
%\base{19.5cm}\\ \\
%\textbf{Nombre}: \base{6.5cm} \textbf{carrera}: \base{5.4cm} \textbf{Calificación}: \base{2cm} \\
%\base{19.5cm}
%cuerpo del documento

%
%\begin{center}
%\textbf{Recomendaciones }
%\end{center}
%\begin{enumerate}[$\bullet$]
%\item No se resuelven preguntas del contenido a evaluar. No se permite el uso del celular, ni compartir mensajes
%\item El uso del equipo de computo es personal e intransferible.
%\item \underline{Tiene una duración máxima de $100$
% \textit{minutos}}
%\end{enumerate}
%\base{19.5cm}

%\begin{center}
%\shadowbox{Estadistica}
%\end{center}

\section{Interpolación}
En general, el problema de la interpolación consiste en determinar una aproximación $f(x)$ en un punto $x_{i}$ del dominio de $f(x)$, a partir del conjunto  ${(x_{i},y_{i})}$ de valores conocidos o en sus vecindades\\
Particularmente, la interpolación polinómica consiste en determinar $f(x_{i})$ a partir de un polinomio $P(x)$ de interpolación de grado menor o igual que $n$ que pasa por los $n+1$ puntos

\begin{enumerate}

\item
Dados los $n+1$ puntos distintos $(x_{i},y_{i})$ el polinomio interpolante que incluye a todos los puntos es único\\
\textbf{Solución}\\
Sean $P(x)$ y $Q(x)$ dos polinomios interpolante de grado menor o igual que $n$, que pasa por los $n+1$ puntos distintos. Luego para todo $i=1,2,...,n+1$ se tiene que:
\begin{center}
	$P(x_{i})=y_{i}$\\
	$Q(x_{i})=y_{i}$
\end{center}
Sea $h(x)=P(x)-Q(x)$ una función polinómica de grado menor o igual que $n$. Donde para los $x_{i}$ se tiene que: $h(x_{i})=P(x_{i})-Q(x_{i})$ como se tiene que $P(x)$ y $Q(x)$ son dos polinomios distintos interpolantes entonces, $h(x_{i})=y_{i}-y_{i}=0$ para todo $i=1,2,...,n+1$ es decir que la función polinómica $h(x)$ de grado menor o igual que $n$ tiene $n+1$ raíces, lo que contradice el \textbf{teorema fundamental del Algebra} Por lo tanto, $h(x)$ debe ser el polinomio cero, por lo tanto, $\forall x_{i}, P(x_{i})-Q(x_{i})=0$ luego,$ P(x_{i})=Q(x_{i}) \forall i=1,2,...,n+1$


\item
Considere el comportamiento de gases no ideales se describe a menudo con la ecuación virial de estado. los siguientes datos para el nitrógeno $N_{2}$


\begin{center}
	\begin{tabular}{|r|r|r|r|r|r|r|r|}
		\hline
		\rowcolor{blue} {T(K)} & 100 & 200 & 300 & 400 & 450 & 500 & 600\\
		\hline
		\rowcolor[gray]{0.9} $B(cm^{3})/mol$ & -160 & -35 & -4.2 & 9.0 &  & 16.9 & 21.3 \\
		
		\hline
	\end{tabular}
\end{center}
Donde T es la temperatura $[K]$ y B es el segundo coeficiente virial.\\
El comportamiento de gases no ideales se describe a menudo con la ecuación virial de estado
\begin{equation}
\dfrac{PV}{RT}=1+\dfrac{B}{V}+\dfrac{C}{V^{2}}+....,
\end{equation}
Donde P es la presión, V el volumen molar del gas, T es la temperatura Kelvin y R es la constante de gas ideal.Los coeficientes $B = B(T)$, $C =
C(T)$, son el segundo y tercer coeficiente virial, respectivamente. En la práctica se usa la serie truncada para aproximar
\begin{equation}
\dfrac{PV}{RT}=1+\dfrac{B}{V}
\end{equation}
En la siguiente figura se muestra como se distribuye la variable $B$ a lo largo de la temperatura
\begin{center} 
	\begin{figure}[h]
		\centering%
		\includegraphics[width=7cm,height=5cm]{I1}
		\caption{Comportamiento del $N_{2}$} 
	\end{figure}
\end{center}

\begin{enumerate}
	\item 
	Determine un polinomio interpolante para este caso:\\
	Teniendo en cuenta que se tomaron cinco puntos el polinomio resultante es
	\begin{equation}
	-520.1 + 5.406917x - 0.02174708x^{2} + 3.955833e-05x^{3} -  
	2.679167e-08x^{4} 
	\end{equation}
	\begin{center} 
		\begin{figure}[h]
			\centering%
			\includegraphics[width=7cm,height=5cm]{I2}
			\caption{Ajuste Polinómico del $N_{2}$} 
		\end{figure}
	\end{center}
	\item
	Utilizando el resultado anterior calcule el segundo y tercer coeficiente virial a 450K.\\
	Para responder
	la pregunta, usando interpolación polinomial, construimos un
	polinomio P que pase por los seis puntos y luego se evalua en 450
	\item
	Grafique los puntos y el polinomio que ajusta
	\item
	Utilice la interpolación de Lagrange y escriba el polinomio interpolante 
	\item
	Grafique los puntos y el polinomio interpolante de Lagrange 
	\item
	¿Cuál es el segundo y tercer coeficiente virial a 450K?. con el método de Lagrange
	\item
	Compare su resultado con la serie truncada (modelo teórico), cuál de las tres aproximaciones es mejor por qué?
	
\end{enumerate}


¿Cuál es el segundo coeficiente virial a 450K?.  de la tabla (ya veremos
cómo), tal y como se muestra en la figura (I.1)
	
 
 
 
 

 
    
            
   
      
     
     	
 	 
  
  





 
 
 	
\end{enumerate}




	



\end{document}
%NOTA:
%GUARDAR EN UN ARCHIVO, EN UNA CARPETA LLAMADA PARCIAL_ESTADISTICA: CON SU NOMBRE COMPLETO
%\vspace{0.1in}
%--------------------------------------------------------------------------------------------------------------------%

%--------------------------------------------------------------------------------------------------------------------%

%\vspace{0.1in}
%--------------------------------------------------------------------------------------------------------------------%

%--------------------------------------------------------------------------------------------------------------------%


